\documentclass{article}
\usepackage[utf8]{inputenc}
\usepackage{amsfonts,latexsym,amsthm,amssymb,amsmath,amscd,euscript}
\usepackage{mathtools}
\usepackage{framed}
% Descomentar fullpage cuando se quiera utilizar menos margen horizontal
%\usepackage{fullpage}
\usepackage{hyperref}
    \hypersetup{colorlinks=true,citecolor=blue,urlcolor =black,linkbordercolor={1 0 0}}

\newenvironment{statement}[1]{\smallskip\noindent\color[rgb]{1.00,0.00,0.50} {\bf #1.}}{}
\allowdisplaybreaks[1]

% Comandos para teoremas, definiciones, ejemplos, lemas, etc. para sus respectivos body types.
\renewcommand*{\proofname}{Prueba}
\renewcommand{\contentsname}{Contenido}

\newtheorem{theorem}{Teorema}
\newtheorem*{proposition}{Proposici\'on}
\newtheorem{lemma}[theorem]{Lema}
\newtheorem{corollary}[theorem]{Corolario}
\newtheorem{conjecture}[theorem]{Conjetura}
\newtheorem*{postulate}{Postulado}
\theoremstyle{definition}
\newtheorem{defn}[theorem]{Definici\'on}
\newtheorem{example}[theorem]{Ejemplo}

\theoremstyle{remark}
\newtheorem*{remark}{Observaci\'on}
\newtheorem*{notation}{Notaci\'on}
\newtheorem*{note}{Nota}

% Define tus comandos para hacer la vida más fácil.
\newcommand{\BR}{\mathbb R}
\newcommand{\BC}{\mathbb C}
\newcommand{\BF}{\mathbb F}
\newcommand{\BQ}{\mathbb Q}
\newcommand{\BZ}{\mathbb Z}
\newcommand{\BN}{\mathbb N}

\title{MAT234 Ecuaciones Diferenciales Ordinarias}
\author{Manuel Loaiza Vasquez}
\date{Abril 2021}

\begin{document}

\maketitle

\vspace*{-0.25in}
\centerline{Pontificia Universidad Cat\'olica del Per\'u}
\centerline{Lima, Per\'u}
\centerline{\href{mailto:manuel.loaiza@pucp.edu.pe}{{\tt manuel.loaiza@pucp.edu.pe}}}
\vspace*{0.15in}

\begin{framed}
  Esta es una plantilla en \LaTeX. Util\'izala en tus listas de ejercicios, notas de clases y mucho m\'as.
\end{framed}

\begin{statement}{1}
  Use la definici\'on de matriz exponencial para probar las propiedades b\'asicas
  de la siguientes proposici\'on dada en clase:
\end{statement}

\begin{proposition}
  Sea $A \in \BC^{n \times n}$. Se cumplen:
  \begin{enumerate}
    \item Si $\Theta$ es la matriz nula, entonces $e^{\Theta} = I$.
    \item $A^m e^A = e^A A^m$, para todo $m \in \BZ^+$.
    \item $(e^A)^T = e^{A^T}$.
    \item Si $AB = BA$, entonces $A e^B = e^B A$ y $e^A e^B = e^B e^A$.
  \end{enumerate}
\end{proposition}

\begin{proof}
  \begin{enumerate}
    \item Desarrollemos $e^{\Theta} = I + \sum_{i = 1}^{\infty} \frac{\Theta^i}{i!}
      = I + \lim_{n \to \infty} \sum_{j = 0}^n \frac{\Theta^{j + 1}}{(j + 1)!}$.
      La sucesi\'on de sumas parciales es siempre igual a cero y por l\'imite de una
      sucesi\'on constante obtenemos $e^{\Theta} = I$.
    \item Analicemos el caso $A = \Theta$ utilizando el inciso anterior:
      $\Theta^m e^\Theta = \Theta I = I \Theta = e^{\Theta} \Theta^m$.
      Ahora procedamos a realizar la prueba por inducci\'on:
      Para $A \neq \Theta, A e^A = A (\lim_{n \to \infty} \sum_{i = 0}^n \frac{A^i}{i!}) =
      \lim_{n \to \infty} \sum_{i = 0}^n \frac{A^{i + 1}}{i!} =
      \lim_{n \to \infty} (\sum_{i = 0}^n \frac{A^i}{i!} A) =
      (\lim_{n \to \infty} \sum_{i = 0}^n \frac{A^i}{i!}) A =
      e^A A$ por aritm\'etica de l\'imites.
      Supongamos que se cumple para $m$, luego
      $A^{m + 1} e^A = A (A^m e^A) = A (e^A A^m) = (A e^A) A^m = (e^A A) A^m = e^A (A A^m) = e^A A^{m + 1}$.
    \item Primero probemos por inducci\'on que $(A^n)^T = (A^T)^n$. Para $n = 1$ es trivial.
      Supongamos que se cumple para $n$. Luego $(A^{n + 1})^T = (A^n A)^T = A^T (A^n)^T = A^T (A^T)^n = (A^T)^{n + 1}$.
      Ya que la transpuesta es continua, por l\'imites y continuidad tenemos
      $(e^A)^T = (\lim_{n \to \infty} \sum_{i = 0}^n \frac{A^i}{i!})^T = \lim_{n \to \infty} (\sum_{i = 0}^n \frac{A^i}{i!})^T$
      Asimismo, como $(A + B)^T = A^T + B^T$, inductivamente podemos extenderlo a una suma finita y utilizando lo probado inicialmente conseguimos
      \begin{align*}
        (e^A)^T &= \lim_{n \to \infty} \left(\sum_{i = 0}^n \frac{A^i}{i!}\right)^T\\
        &= \lim_{n \to \infty} \sum_{i = 0}^n \frac{(A^i)^T}{i!}\\
        &= \lim_{n \to \infty} \sum_{i = 0}^n \frac{(A^T)^i}{i!}\\
        &= e^{A^T}.
      \end{align*}
    \item Desarrollamos la primera expresi\'on:
      \begin{align*}
        A e^B &= A \left(\sum_{i = 0}^{\infty} \frac{B^i}{i!}\right)\\
        &= \left(\sum_{i = 0}^{\infty} A \frac{B^i}{i!}\right)\\
        &= \left(\sum_{i = 0}^{\infty} \frac{B^i}{i!} A\right)\\
        &= \left(\sum_{i = 0}^{\infty} \frac{B^i}{i!}\right) A\\
        &= e^B A.
      \end{align*}
      Ahora desarrollamos la segunda expresi\'on:
      \begin{align*}
        e^A e^B &= \left(\sum_{i = 0}^{\infty} \frac{A^i}{i!}\right)\left(\sum_{j = 0}^{\infty} \frac{B^j}{j!}\right)\\
        &= \sum_{i = 0}^{\infty} \sum_{j = 0}^{\infty} \frac{A^i B^j}{i! j!}\\
        &= \sum_{j = 0}^{\infty} \sum_{i = 0}^{\infty} \frac{B^j A^i}{j! i!}\\
        &= \left(\sum_{j = 0}^{\infty} \frac{B^j}{j!}\right)\left(\sum_{i = 0}^{\infty} \frac{A^i}{i!}\right)\\
        &= e^B e^A.
      \end{align*}
  \end{enumerate}
\end{proof}

\end{document}

