\documentclass{article}
\usepackage[utf8]{inputenc}
\usepackage{amsfonts,latexsym,amsthm,amssymb,amsmath,amscd,euscript}
\usepackage{mathtools}
\usepackage{framed}
% Descomentar fullpage cuando se quiera utilizar menos margen horizontal
%\usepackage{fullpage}
\usepackage{hyperref}
    \hypersetup{colorlinks=true,citecolor=blue,urlcolor =black,linkbordercolor={1 0 0}}
\usepackage{cleveref}
\newenvironment{statement}[1]{\smallskip\noindent\color[rgb]{1.00,0.00,0.50} {\bf #1.}}{}
\allowdisplaybreaks[1]

% Comandos para teoremas, definiciones, ejemplos, lemas, etc. para sus respectivos body types.
\renewcommand*{\proofname}{Prueba}
\renewcommand{\contentsname}{Contenido}

\newtheorem{theorem}{Teorema}
\newtheorem*{proposition}{Proposici\'on}
\newtheorem{lemma}[theorem]{Lema}
\newtheorem{corollary}[theorem]{Corolario}
\newtheorem{conjecture}[theorem]{Conjetura}
\newtheorem*{postulate}{Postulado}
\theoremstyle{definition}
\newtheorem{defn}[theorem]{Definici\'on}
\newtheorem{example}[theorem]{Ejemplo}

\theoremstyle{remark}
\newtheorem*{remark}{Observaci\'on}
\newtheorem*{notation}{Notaci\'on}
\newtheorem*{note}{Nota}

\usepackage{listings}
\usepackage[usenames,dvipsnames]{color}

% Keywords command
\providecommand{\keywords}[1]
{
  \bigskip
  \textbf{Keywords} #1
}

\crefname{lemma}{Lema}{Lemas}

% Define tus comandos para hacer la vida más fácil.
\newcommand{\BR}{\mathbb R}
\newcommand{\BC}{\mathbb C}
\newcommand{\BF}{\mathbb F}
\newcommand{\BQ}{\mathbb Q}
\newcommand{\BZ}{\mathbb Z}
\newcommand{\BN}{\mathbb N}

\title{MAT234 Ecuaciones Diferenciales Ordinarias}
\author{Manuel Loaiza Vasquez}
\date{Abril 2021}

\begin{document}

\maketitle

\vspace*{-0.25in}
\centerline{Pontificia Universidad Cat\'olica del Per\'u}
\centerline{Lima, Per\'u}
\centerline{\href{mailto:manuel.loaiza@pucp.edu.pe}{{\tt manuel.loaiza@pucp.edu.pe}}}
\vspace*{0.15in}

\begin{framed}
  Esta es una plantilla en \LaTeX. Util\'izala en tus listas de ejercicios, notas de clases y mucho m\'as.
\end{framed}

\begin{statement}{1}
  Use la definici\'on de matriz exponencial para probar las propiedades b\'asicas
  de la siguientes proposici\'on dada en clase:
\end{statement}

\begin{proposition}
  Sea $A \in \BC^{n \times n}$. Se cumplen:
  \begin{enumerate}
    \item Si $\Theta$ es la matriz nula, entonces $e^{\Theta} = I$.
    \item $A^m e^A = e^A A^m$, para todo $m \in \BZ^+$.
    \item $(e^A)^T = e^{A^T}$.
    \item Si $AB = BA$, entonces $A e^B = e^B A$ y $e^A e^B = e^B e^A$.
  \end{enumerate}
\end{proposition}

\begin{proof}
  \begin{enumerate}
    \item Desarrollemos $e^{\Theta} = I + \sum_{i = 1}^{\infty} \frac{\Theta^i}{i!}
      = I + \lim_{n \to \infty} \sum_{j = 0}^n \frac{\Theta^{j + 1}}{(j + 1)!}$.
      La sucesi\'on de sumas parciales es siempre igual a cero y por l\'imite de una
      sucesi\'on constante obtenemos $e^{\Theta} = I$.
    \item Analicemos el caso $A = \Theta$ utilizando el inciso anterior:
      $\Theta^m e^\Theta = \Theta I = I \Theta = e^{\Theta} \Theta^m$.
      Ahora procedamos a realizar la prueba por inducci\'on:
      Para $A \neq \Theta, A e^A = A (\lim_{n \to \infty} \sum_{i = 0}^n \frac{A^i}{i!}) =
      \lim_{n \to \infty} \sum_{i = 0}^n \frac{A^{i + 1}}{i!} =
      \lim_{n \to \infty} (\sum_{i = 0}^n \frac{A^i}{i!} A) =
      (\lim_{n \to \infty} \sum_{i = 0}^n \frac{A^i}{i!}) A =
      e^A A$ por aritm\'etica de l\'imites.
      Supongamos que se cumple para $m$, luego
      $A^{m + 1} e^A = A (A^m e^A) = A (e^A A^m) = (A e^A) A^m = (e^A A) A^m = e^A (A A^m) = e^A A^{m + 1}$.
    \item Primero probemos por inducci\'on que $(A^n)^T = (A^T)^n$. Para $n = 1$ es trivial.
      Supongamos que se cumple para $n$. Luego $(A^{n + 1})^T = (A^n A)^T = A^T (A^n)^T = A^T (A^T)^n = (A^T)^{n + 1}$.
      Ya que la transpuesta es continua, por l\'imites y continuidad tenemos
      $(e^A)^T = (\lim_{n \to \infty} \sum_{i = 0}^n \frac{A^i}{i!})^T = \lim_{n \to \infty} (\sum_{i = 0}^n \frac{A^i}{i!})^T$
      Asimismo, como $(A + B)^T = A^T + B^T$, inductivamente podemos extenderlo a una suma finita y utilizando lo probado inicialmente conseguimos
      \begin{align*}
        (e^A)^T &= \lim_{n \to \infty} \left(\sum_{i = 0}^n \frac{A^i}{i!}\right)^T\\
        &= \lim_{n \to \infty} \sum_{i = 0}^n \frac{(A^i)^T}{i!}\\
        &= \lim_{n \to \infty} \sum_{i = 0}^n \frac{(A^T)^i}{i!}\\
        &= e^{A^T}.
      \end{align*}
    \item Desarrollamos la primera expresi\'on:
      \begin{align*}
        A e^B &= A \left(\sum_{i = 0}^{\infty} \frac{B^i}{i!}\right)\\
        &= \left(\sum_{i = 0}^{\infty} A \frac{B^i}{i!}\right)\\
        &= \left(\sum_{i = 0}^{\infty} \frac{B^i}{i!} A\right)\\
        &= \left(\sum_{i = 0}^{\infty} \frac{B^i}{i!}\right) A\\
        &= e^B A.
      \end{align*}
      Ahora desarrollamos la segunda expresi\'on:
      \begin{align*}
        e^A e^B &= \left(\sum_{i = 0}^{\infty} \frac{A^i}{i!}\right)\left(\sum_{j = 0}^{\infty} \frac{B^j}{j!}\right)\\
        &= \sum_{i = 0}^{\infty} \sum_{j = 0}^{\infty} \frac{A^i B^j}{i! j!}\\
        &= \sum_{j = 0}^{\infty} \sum_{i = 0}^{\infty} \frac{B^j A^i}{j! i!}\\
        &= \left(\sum_{j = 0}^{\infty} \frac{B^j}{j!}\right)\left(\sum_{i = 0}^{\infty} \frac{A^i}{i!}\right)\\
        &= e^B e^A.
      \end{align*}
  \end{enumerate}
\end{proof}

\begin{statement}{2}
  Demuestre que $e^{c I  + A} = e^c e^A$, para todos los escalares $c$ y todas las matrices cuadradas $A$.
\end{statement}

\begin{proof}
  Tenemos que $(c I) A = c (I A) = c (A I) = (A I) c = A (I c) = A (c I)$ conmutan,
  por lo que $e^{c I + A} = e^{c I} e^A$. Asimismo, tenemos
  \begin{align*}
    e^{cI} &= \sum_{i = 0}^{\infty} \frac{(c I)^i}{i!}\\
    &= \sum_{i = 0}^{\infty} \frac{c^i I}{i!}\\
    &= \left(\sum_{i = 0}^{\infty} \frac{c^i}{i!}\right) I\\
    &= e^c I.
  \end{align*}
  De este modo, al utilizar la \'ultima igualdad logramos
  \begin{align*}
    e^{c I + A} &= e^{c I} e^A\\
    &= (e^c I) e^A\\
    &= e^c (I e^A)\\
    &= e^c e^A.
  \end{align*}
\end{proof}

\begin{statement}{3}
  Si $A^2 = A$, encuentre una f\'ormula para $e^A$.
\end{statement}

\begin{proof}
  Probemos el siguiente lema para proceder:
  \begin{lemma}\label{lem01}
    Si $A^2 = A$, entonces $A^n = A$, para todo $n \geq 2$.
  \end{lemma}
  \begin{proof}
    Procedamos a realizar la prueba por inducci\'on.
    Para $n = 2$ nos lo han dado de regalo.
    Supongamos que se cumple para $n \geq 2$.
    Luego $A^{n + 1} = A^n A = A A = A^2 = A$.
  \end{proof}
  Finalmente, desarollemos $e^A$ y utilicemos \cref{lem01}
  \begin{align*}
    e^A &= I + \sum_{i = 1}^{\infty} \frac{A^i}{i!}\\
    &= I + \sum_{i = 1}^{\infty} \frac{A}{i!}\\
    &= I + A \sum_{i = 1}^{\infty} \frac{1}{i!}\\
    &= I + A (e - 1).
  \end{align*}
\end{proof}

\begin{statement}{4}
  Calcule $e^A$ para las matrices
\end{statement}

\begin{enumerate}
  \item
    $ A = \begin{pmatrix} a & b \\ 0 & a \end{pmatrix}$:
    Sea $B = \begin{pmatrix}0 & b \\ 0 & 0\end{pmatrix}$,
    reescribamos $A = a I + B$.
    Utilizando el resultado del segundo ejercicio tenemos
    $
      e^A = e^{aI + B} = e^a e^B.
    $
    Asimismo, $B^2 = \Theta$, por lo que
    \[
      e^A = e^a \sum_{i = 0}^{\infty} \frac{B^i}{i!} = e^a (I + B) =
      \begin{pmatrix} e^a & b e^a \\ 0 & e^a \end{pmatrix}.
    \]
  \item
    $ A = \begin{pmatrix} a & b \\ 0 & 0 \end{pmatrix}$:
    Probemos que $A^n = a^{n - 1} A$ para $n \geq 1$ utilizando inducci\'on.
    Para $n = 1$ es trivial. Supongamos que se cumple para $n$. As\'i
    $
    A^{n + 1} = A^n A = a^{n - 1} A A =
    a^{n - 1} \begin{pmatrix} a^2 & ab \\ 0 & 0 \end{pmatrix} =
    a^n \begin{pmatrix} a & b \\ 0 & 0 \end{pmatrix} =
    a^n A.
    $
    Finalmente,
    \begin{align*}
      e^A &= I + \sum_{i = 1}^{\infty} \frac{A^i}{i!}\\
      &= I + A \sum_{i = 1}^{\infty} \frac{a^{i - 1}}{i!}\\
      &= I + \frac{1}{a} A \sum_{i = 1}^{\infty} \frac{a^i}{i!}\\
      &= I + \frac{1}{a} A (e^a - 1)\\
      &= \begin{pmatrix} e^a & \frac{b}{a}(e^a - 1) \\ 0 & 1\end{pmatrix}.
    \end{align*}
  \item
    $A = \begin{pmatrix} a & 0 \\ b & 0 \end{pmatrix}$:
    F\'acilmente observamos que la matriz dada es la transpuesta del inciso anterior, por lo que
    \[
      e^A = \begin{pmatrix} e^a & 0 \\ \frac{b}{a}(e^a - 1) & 1 \end{pmatrix}.
    \]
\end{enumerate}

\begin{statement}{5}
  Si $A^2 = I$, demuestre que
  \[
    2 e^A = (e + \frac{1}{e}) I + (e - \frac{1}{e}) A.
  \]
\end{statement}

\begin{proof}
  Desarrollamos
  \begin{align*}
    e^A &= \sum_{i = 0}^{\infty} \frac{A^i}{i!}\\
    &= \sum_{k = 0}^{\infty} \frac{A^{2k}}{(2k)!} + \sum_{k = 0}^{\infty} \frac{A^{2k + 1}}{(2k + 1)!}\\
    &= \sum_{k = 0}^{\infty} \frac{I}{(2k)!} + \sum_{k = 0}^{\infty} \frac{A}{(2k + 1)!}\\
    &= \left(\sum_{k = 0}^{\infty} \frac{1}{(2k)!}\right) I + \left(\sum_{k = 0}^{\infty} \frac{1}{(2k + 1)!}\right) A\\
    &= \cosh(1) I + \sinh(1) A\\
    &= \left(\frac{e + e^{-1}}{2}\right) I + \left(\frac{e - e^{-1}}{2}\right) A\\
    2 e^A &= \left(e + \frac{1}{e}\right) I + \left(e - \frac{1}{e}\right) A.
  \end{align*}
\end{proof}

\begin{statement}{6}
  Supongamos que $\lambda \in \BC$ y $x \in \BC^n$ es no nulo tal que $Ax = \lambda x$.
  Demuestre que $e^A x = e^{\lambda} x$.
\end{statement}

\begin{proof}
  Por definici\'on
  \begin{align*}
    e^A x &= \left(\sum_{i = 0}^{\infty} \frac{A^i}{i!}\right) x\\
    &= \sum_{i = 0}^{\infty} \frac{A^i x}{i!}\\
    &= \sum_{i = 0}^{\infty} \frac{\lambda^i x}{i!}\\
    &= \left(\sum_{i = 0}^{\infty} \frac{\lambda^i}{i!}\right) x\\
    &= e^{\lambda} x.
  \end{align*}
\end{proof}

\end{document}
