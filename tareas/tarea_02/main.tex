\documentclass{article}
\usepackage[utf8]{inputenc}
\usepackage{amsfonts,latexsym,amsthm,amssymb,amsmath,amscd,euscript}
\usepackage{mathtools}
\usepackage{framed}
% Descomentar fullpage cuando se quiera utilizar menos margen horizontal
%\usepackage{fullpage}
\usepackage{hyperref}
    \hypersetup{colorlinks=true,citecolor=blue,urlcolor =black,linkbordercolor={1 0 0}}
\usepackage{cleveref}
\newenvironment{statement}[1]{\smallskip\noindent\color[rgb]{1.00,0.00,0.50} {\bf #1.}}{}
\allowdisplaybreaks[1]

% Comandos para teoremas, definiciones, ejemplos, lemas, etc. para sus respectivos body types.
\renewcommand*{\proofname}{Prueba}
\renewcommand{\contentsname}{Contenido}

\newtheorem{theorem}{Teorema}
\newtheorem*{proposition}{Proposici\'on}
\newtheorem{lemma}[theorem]{Lema}
\newtheorem{corollary}[theorem]{Corolario}
\newtheorem{conjecture}[theorem]{Conjetura}
\newtheorem*{postulate}{Postulado}
\theoremstyle{definition}
\newtheorem{defn}[theorem]{Definici\'on}
\newtheorem{example}[theorem]{Ejemplo}

\theoremstyle{remark}
\newtheorem*{remark}{Observaci\'on}
\newtheorem*{notation}{Notaci\'on}
\newtheorem*{note}{Nota}

\usepackage{listings}
\usepackage[usenames,dvipsnames]{color}

% Keywords command
\providecommand{\keywords}[1]
{
  \bigskip
  \textbf{Keywords} #1
}

\crefname{lemma}{Lema}{Lemas}

% Define tus comandos para hacer la vida más fácil.
\newcommand{\BR}{\mathbb R}
\newcommand{\BC}{\mathbb C}
\newcommand{\BF}{\mathbb F}
\newcommand{\BQ}{\mathbb Q}
\newcommand{\BZ}{\mathbb Z}
\newcommand{\BN}{\mathbb N}
\DeclareMathOperator{\Tr}{Tr}

\title{MAT234 Ecuaciones Diferenciales Ordinarias}
\author{Manuel Loaiza Vasquez}
\date{Abril 2021}

\begin{document}

\maketitle

\vspace*{-0.25in}
\centerline{Pontificia Universidad Cat\'olica del Per\'u}
\centerline{Lima, Per\'u}
\centerline{\href{mailto:manuel.loaiza@pucp.edu.pe}{{\tt manuel.loaiza@pucp.edu.pe}}}
\vspace*{0.15in}

\begin{framed}
  Lista de ejercicios relacionada al material cubierto en las semanas 3 - 5 con
  el profesor Ruben Agapito Ruiz.
\end{framed}

\begin{statement}{1}
  Realice lo siguiente:
\end{statement}

\begin{statement}{a}
  Pruebe que $f(x) = |x|^{1 / 2}$ no es localmente lipschitziana en $0$.
  Esto quiere decir que $f$ no es Lipschitz en cualquier intervalo $(a, b)$ que
  contenga al origen.
  Asimismo, $f$ es Lipschitz en cualquier intervalo (finito o infinito) lejos
  del origen.
  ¿Es $f$ lipschitziana en $(0, 1)$?
\end{statement}

\begin{statement}{b}
  Escriba tres soluciones diferentes para
  \[
    x'(t) = \sqrt{|x(t)|}
  \]
  que satisfagan $x(0) = 0$.
\end{statement}

\begin{statement}{2}
  Considere la ecuaci\'on de Bernoulli
  \[
    x' + \varphi(t) x = \psi(t) x^n,
  \]
  donde $\phi, \psi$ son funciones continuas.
  Para $n \neq 1$, la EDO es no lineal.
  Pruebe que la ecuaci\'on puede ser reducida a una EDO lineal v\'ia la
  sustituci\'on $y = x^{1 - n}$. Encuentre la soluci\'on general de
  \[
    x' + t^n x = x^n.  
  \]
\end{statement}

\begin{statement}{3}
  Considere la ecuaci\'on de Jacobi
  \[
    (a_1 + b_1 t + c_1 x) (t \, dx - x \, dt) - (a_2 + b_2 t + c_2 x) dx + (a_3 + b_3 t + c_3 x) dt = 0,  
  \]
  donde $a_i, b_i, c_i$ son constantes, para todo $i$ desde $1$ hasta $3$.
  Transforme las variables $(t, x) \to (\tau, y)$, donde
  $t = \tau + \alpha$ y $x = y + \beta$.
  Escoja $\alpha, \beta$ apropiadamente para obtener la ecuaci\'on
  \[
    \tau dy - y d \tau   + \varphi\left(\frac{y}{\tau}\right) dy +
    \psi\left(\frac{y}{\tau}\right) d\tau = 0.
  \]
  Ahora realice la sustituci\'on $y = \tau u$ para convertirla a una ecuaci\'on
  de Bernoulli
  \[
    \frac{d \tau}{du} + h(u) \tau + g(u) \tau^2 = 0.
  \]
\end{statement}

\begin{statement}{4}
  Considere la ecuaci\'on generalizada de Riccati
  \[
    y' + \psi(t) y^2 + \varphi(t) y + \xi(t) = 0,
  \]
  donde $\psi, \varphi, \xi$ son funciones continuas.
  En general, uno no tiene soluciones en forma expl\'icita pero asumamos que
  $x_1(t)$ es una soluci\'on conocida. Sea $x(t)$ cualquier otra soluci\'on.
  Escribimos $x = x_1 + y$. Muestre que $y$ satisface la ecuaci\'on de Bernoulli
  \[
    y' + (2 x_1 \psi + \varphi) y + \psi y^2 = 0.  
  \]
\end{statement}

\begin{statement}{5}
  Encuentre la soluci\'on general de la ecuaci\'on
  \[
    y' + y^2 + y - (1 + t + t^2) = 0.  
  \]
\end{statement}

\begin{statement}{6}
  Supongamos que $\psi(t) \neq 0$ para todo $t$ en la ecuaci\'on de Riccati
  \[
    y' + \psi(t) y^2 + \varphi(t) y + \xi(t) = 0,
  \]
  Muestre que la transformaci\'on $x = z' / (\psi z)$ reduce la ecuaci\'on
  de Riccati a una ecuaci\'on lineal de segundo orden para $z$.
\end{statement}

\begin{proof}
  Sea $x = \psi y$. Derivamos respecto a $t$
  \begin{align*}
    x' &= \psi' y + \psi y'\\
    &= \psi' y + \psi(-\psi y^2 - \varphi y - \xi)\\
    &= \psi' y - \psi^2 y^2 - \psi \varphi y - \psi \xi\\
    &= \frac{\psi' x}{\psi} - \frac{\psi^2 x^2}{\psi^2} - \frac{\psi \varphi x}{\psi} - \psi \xi\\
    &= -x^2 - \varphi x - \frac{\psi' x}{\psi} - \psi \xi'\\
    &= -x^2 - \left(\varphi + \frac{\psi'}{\psi}\right) x - \psi \xi.
  \end{align*}
  Ahora hacemos $x = z' / z$. Primero derivamos esto respecto a $t$ para conseguir
  \begin{align*}
    x' &= \frac{z'' z - z' z'}{z^2}\\
    &= \frac{z''}{z} - \left(\frac{z'}{z}\right)^2\\
    &= \frac{z''}{z} - x^2
  \end{align*}
  y reemplazamos esto en nuestra expresi\'on para obtener
  \begin{align*}
    x' + x^2 + \left(\varphi + \frac{\psi'}{\psi}\right) x + \psi \xi = 0\\
    \frac{z''}{z} + \left(\varphi + \frac{\psi'}{\psi}\right) \frac{z'}{z} + \psi \xi = 0\\
    z'' + \left(\varphi + \frac{\psi'}{\psi}\right) z' + \psi \xi z = 0.
  \end{align*}
  lo cual es una ecuaci\'on lineal de segundo orden.
\end{proof}

\begin{statement}{7}
  Reduzca la ecuaci\'on original de Riccati
  \[
    y' + a y^2 = b t^m,  
  \]
  donde $a$ y $b$ son constantes, a una EDO lineal de segundo orden
  \[
    z'' - a b t^m z = 0  
  \]
  utilizando la transformaci\'on de Ejercicio 6.
\end{statement}

\begin{statement}{8}
  Tres soluciones de una EDO lineal no homog\'enea de segundo orden en $\BR$ son
  \[
    \varphi_1(t) = t^2,
    \varphi_2(t) = t^2 + e^{2t},
    \varphi_3(t) = 1 + t^2 + 2e^{2t}.  
  \]
  Encuentre la soluci\'on general de esta ecuaci\'on.
\end{statement}

\begin{statement}{9}
  Resuelva el PVI
  \begin{gather*}
    y'' + y = f(x),\\
    y(0) = 0,\\
    y'(0) = 1,\\
  \end{gather*}
  donde
  \[
    f(x) =
    \begin{cases}
      0 & x < 0,\\
      \cos x & 0 \leq x \leq 4 \pi,\\
      0 & x > 4 \pi.
    \end{cases}  
  \]
  v\'ia la construcci\'on de una funci\'on de Green.
\end{statement}

\begin{statement}{10}
  (Ecuaci\'on de Cauchy-Euler)
  Resuelva el PVF
  \begin{gather*}
    x^2 y'' - 4xy' + 6y = x^4,\\
    y(1) - y'(1) = 0,\\
    y(3) = 0
  \end{gather*}
  v\'ia la construcci\'on de una funci\'on de Green.
\end{statement}

\end{document}
